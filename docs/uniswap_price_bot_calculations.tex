\documentclass[a4paper, 11pt]{article}

\usepackage{mathtools}
\usepackage{amsfonts}
\usepackage{amsmath}
\usepackage{parskip}
\usepackage[numbers]{natbib}
%\usepackage[round,authoryear]{natbib}
\usepackage{breqn}
\usepackage[section]{placeins}
\usepackage{float}
\usepackage[]{graphicx, caption}
\usepackage[urlcolor=blue]{hyperref}
\usepackage{background}
\usepackage{url}
\usepackage[nottoc,numbib]{tocbibind}
\usepackage[utf8]{inputenc}
\usepackage[]{color}
\usepackage[]{xcolor}
\usepackage{hyperref}
\usepackage{bm}

\usepackage[english]{babel}

\usepackage{subcaption}
\usepackage[labelformat=parens,labelsep=quad,skip=3pt]{caption}

% in preamble
\usepackage{movie15}
% in documenet

\tolerance=1
\emergencystretch=\maxdimen
\hyphenpenalty=10000
\hbadness=10000

\definecolor{googleblue800}{RGB}{2,119,189}
\definecolor{googlegrey100}{RGB}{245,245,245}

\usepackage{geometry}

\geometry{
    a4paper,
    total={170mm,253mm},
    left=20mm,
    top=22mm,
}


\hypersetup{%
    pdfborder = {0 0 0},
    colorlinks,
	linkcolor=black,
	citecolor=black,
	urlcolor=blue,
}
\newcommand{\hrefc}[3][blue]{\href{#2}{\color{#1}{#3}}}%

\usepackage{scrpage2}
\ifoot[]{}
\cfoot[]{}
\ofoot[\pagemark]{\pagemark}

\pagestyle{scrplain}

\newcommand{\chapquote}[3]{\begin{quotation} \textit{#1} \end{quotation} \begin{flushright} - #2, \textit{#3}\end{flushright} }

\begin{document}


%\begin{titlepage}
%\backgroundsetup{
%scale=1,
%angle=0,
%opacity=1,  %% adjust
%contents={\includegraphics[width=\paperwidth,height=\paperheight]{./imgs/frontpage.pdf}}
%}
%\phantom{whatever}
%
%
%\end{titlepage}

\newpage

\backgroundsetup{
scale=1,
angle=0,
opacity=1,
color=black,
contents={
\begin{tikzpicture}[remember picture,overlay]
\node at ([xshift=-10.5cm,yshift=1.35cm] current page.south east) % Adjust the position of the logo.
{\includegraphics[scale=0.15]{./imgs/logo.pdf}}; % logo goes here
\end{tikzpicture}}
}

%\tableofcontents
%\newpage

\title{Uniswap price bot calculations}
\author{Michael A. Bentley \\ Euler XYZ}
\maketitle

\section{Introduction}

Here, we consider how to make a swap on a Uniswap v2 pool (or v3 pool with a max-width position) in order to move the price of an asset to some target price. 

\subsection{General mechanics}

Consider a Uniswap pool with two assets, labelled $X$ and $Y$. Let the global reserves in the pool at time $t = 0$ be denoted by $x_0$ and $y_0$, respectively. 

There are two types of events that can alter a Uniswap pool: swap events and liquidity provision events. Both events alter the individual reserves, $x_0$ and $y_0$, but swaps alter the price and leave liquidity unchanged, whilst liquidity provision events leave the price unchanged and alter liquidity. 

Liquidity in the pool $k_0$ is defined by the constant-product formula

\begin{equation}
\label{eq:k_0}
k_0 = x_0 \cdot y_0
\end{equation}

The price of asset $Y$ in terms of asset $X$ at time $t = 0$ is denoted $p^y_0$ and simply given by

\begin{equation}
\label{eq:p^y_0}
p^y_0 = \frac{y_0}{x_0}.
\end{equation}

\subsection{Swaps}

Swaps involve a user trading an amount $\Delta x$ of asset $X$ for an amount $\Delta y$ of asset $Y$, or vice versa. Consider a trade $(\Delta x, \Delta y)$. If the user wants to swap $\Delta x$ amount of $X$, what is the amount $\Delta y$ they get of asset $Y$ in return, or vice versa? 

To answer these questions, first note that the reserves after the swap must obey the constant-product rule, meaning that

\begin{equation}
(x_0 + \Delta x) (y_0 + \Delta y) = k_0.
\end{equation}

Rearranging, we find that the change $\Delta y$ induced by a swap size of $\Delta x$ is therefore

\begin{equation}
\label{eq:delta-y}
\Delta y = \frac{k}{x_0 + \Delta x} - y_0 =  y_0 \cdot \frac{x_0}{x_0 + \Delta x} - y_0 = y_0 \left(  \frac{x_0}{x_0 + \Delta x} - 1 \right) = - y_0 \left(  \frac{\Delta x}{x_0 + \Delta x} \right).
\end{equation}

By symmetry, the change $\Delta x$ induced by a swap of size $\Delta y$ is 

\begin{equation}
\label{eq:delta-x}
\Delta x = - x_0 \left(  \frac{\Delta y}{y_0 + \Delta y} \right).
\end{equation}

The price at time $t = 1$ after a swap is $p^y_1 = (y_0 + \Delta y) / (x_0 + \Delta x) $.

\subsection{Targeting a price}

Now suppose we want to make a swap to target a particular price, $p_*^y$, so that the price after the swap is given by

\begin{equation}
\label{eq:target}
p^y_* 
= 
\frac{y_0 + \Delta y}{x_0 + \Delta x}.
\end{equation}

If the target price is greater than the current price at time $p^y_*  > p^y_0$, then we want to increase $p^y_0$. We can do this by swapping in $\Delta y$ for an amount $\Delta x$. Substituting in to equation \eqref{eq:target} for $\Delta x$ from equation \eqref{eq:delta-x}, we have

\begin{equation}
p^y_* 
= 
\frac{y_0 + \Delta y}{x_0 - x_0 \left(  \frac{\Delta y}{y_0 + \Delta y} \right)}
=
\frac{y_0 + \Delta y}{x_0 \left(  \frac{y_0}{y_0 + \Delta y} \right)}
=
\frac{(y_0 + \Delta y)^2}{x_0 y_0}.
\end{equation}

Solving for $\Delta y$, the amount we need to swap in to drive the price to the target, we obtain

\begin{equation}
\Delta y
=
\sqrt{x_0 y_0 p^y_*} - y_0.
\end{equation}

If the target price is less than the current price at time $p^y_*  < p^y_0$, then we want to decrease $p^y_0$. We can do this by swapping in $\Delta x$ for an amount $\Delta y$. Substituting in to equation \eqref{eq:target} for $\Delta y$ from equation \eqref{eq:delta-y}, we have

\begin{equation}
p^y_* 
= 
\frac{y_0 + \Delta y}{x_0 + \Delta x}
=
\frac{y_0 \left(  \frac{x_0}{x_0 + \Delta x} \right)}{x_0 + \Delta x}
=
\frac{x_0 y_0}{(x_0 + \Delta x)^2}
\end{equation}

Solving for $\Delta x$, the amount we need to swap in to drive the price to the target, we obtain

\begin{equation}
\Delta x
=
\sqrt{\frac{x_0 y_0}{p^y_*}} - x_0.
\end{equation}


\end{document}